%! TEX root = 'main.tex'
\section{Introduction}
\label{sec:implant-introduction}

Critical infrastructure such as power grids comprises physical and cyber systems and assets that vital to national security. Their failure or incapacity would cause a significant impact on people's daily life on a large scale. Since the infrastructures systems are automated and computer-controlled, the industry informatization also brings security concerns. 

The industrial control system (ICS) interconnects and controls the physical production assets.  Compared to traditional IT infrastructures, the physical assets and the computer-based network's interconnection is a unique ICS feature. It is managed through an embedded system known as the programmable logic controller (PLC). In recent years, several worldwide incidents, such as Stuxnet~\cite{langner2011stuxnet}, BlackEnergy~\cite{cherepanov2016blackenergy} targeted critical infrastructures~\cite{case2016analysis}~\cite{soltan2016power}~\cite{zhang2013time}~\cite{williams2016power} operated by ICS. Moreover, to sabotaging physical facilities, PLC is the preferred target of attack.

With the continuous emergence of attacks, protection measures have also been strengthened. ICS security has been traditionally handled using network security practices such as access control. A common strategy is to use an isolated network from the Internet. Through physical access control, less software service is exposed to the public, reducing the potential attack surface, especially vulnerability-exploit-based attacks. 

However, real-world APT attacks show that even an air-gapped network is penetratable~\cite{langner2011stuxnet}. The core of an APT attack is essentially a trojan backdoor that gains access to a computer network and remains undetected for an extended period. The most crucial feature of a trojan is stealthiness. How to stay within the device is an essential issue that sophisticated APT attacks should consider. In particular, firmware modification~\cite{garcia2017hey}~\cite{newman2011scada}~\cite{basnight2013firmware}~\cite{blanco2012one}~\cite{cui2013firmware}~\cite{konstantinou2015impact}~\cite{schulz2017nexmon} is the currently practiced attack plan. It injects malicious code into the target PLC, changes the working logic that runs in the device. However, this attack subject to the firmware verification~\cite{mcminn2012firmware}~\cite{wang2015confirm}~\cite{lee2016binding}~\cite{li2011viper}~\cite{seshadri2004swatt}~\cite{li2010sbap} and update authentication~\cite{lee2017blockchain}~\cite{moran2019firmware}~\cite{choi2016secure} method. It would be much harder for the attacker to implant the malicious code once the firmware update is encrypted and digitally-signed and the system applies the methods mentioned above.


Another critical point for the APT attack is choosing the trigger event. Usually, the PLC has a dedicated real-time microcontroller to control the physical world through its IO pins. However, the microcontroller does not directly communicate with the host, the central control terminal (human-machine interface, HMI). Therefore, firmware modification attacks can perform a preset task individually, but it is difficult to react to PLC firmware updates or coordinate a distributed attack with other controlled nodes, especially among air-gapped networks.

We propose an alternative approach to circumvent existing software mitigations, \name, a parasitical hardware implant inside a PLC attaching to its circuit board and remotely controlled through GSM network.  With the recent emerging concept of supply chain attacks and real-world incidents~\cite{oxfordsolarwinds}, such a hardware attack appears to be more practical. The hardware implant can be pre-installed during the PLC device's assembly line or even during the shipment. We design it to be flexible because the PLC will load operating logic only after being deployed, and it is plausible that the ICS updates PLC's operating logic frequently. The hardware implant is specialized for the device's circuit board, the microcontroller, and other chips.  It controls the IO through the digital signal and bus-level protocol hijacking, independent of the PLC's firmware.

Memory bus and interconnect protocols such as SPI, I2C are all potential targets. Low-speed protocols are prevalent due to their simplicity. For instance, JTAG (Joint Test Action Group) is an industry-standard for verifying designs and testing integrated circuits (IC) after manufacture. 	On ARM microcontrollers, extensive hardware features are also provided through this interface for system-level debugging and tracing. It can read/write registers of processor and memory during system runtime. We leverage this interface for IO controlling purposes, and also, we can fetch the PLC's firmware and operating logic for further offline analysis. Furthermore, with the ability to communicate through a GSM network, it is practical to control multiple nodes and organize a distributed attack simultaneously.


\textbf{Contributions.} To summarize, we make the following contributions in this paper:
\begin{itemize}[leftmargin=*]
	\item We present a novel attack class on industrial control systems: a parasitical hardware implant, which is completely invisible to the ICS control system.
	\item We disassembled and reverse engineered the circuit boards of a widely deployed Allen Bradley 1769-L18ER-BBIB CompactLogix 5370 PLC. 
	\item We develop a prototype implementation of \name, which is a small size device installed inside the Allen Bradley PLC. 
	\item We write a JTAG driver that runs bare-metally on a microcontroller with minimal resource usage.
	\item We test and evaluate \name and conduct a synchronized attack with multiple controlled Allen Bradley PLCs. 
\end{itemize}


\textbf{Roadmap.} The rest of this paper is organized as the following. In~\autoref{sec:implant-background}, we provide the necessary background on programmable logic controllers and JTAG protocol. ~\autoref{sec:implant-overview} describes the objectives, adversary model and scope, challenges, and architecture of \name. ~\autoref{sec:implant-design} describes how we reverse-engineered the Allen Bradley PLC and prototyped \name with implementation details (~\autoref{sec:implant-implementation}) and evaluation (~\autoref{sec:implant-evaluation}), respectively. ~\autoref{sec:implant-relatedwork} provides a review of related work in the area of embedded system firmware attacks and mitigations. Finally~\autoref{sec:implant-conclusions} concludes the paper.


