%! TEX root = 'main.tex'
\section{Introduction}
\label{sec:implant-introduction}

% Importance of ICS
Critical infrastructure such as power grids comprises physical and cyber systems and assets that are vital to national security. Their failure or incapacity would have a significant impact on people's daily life on a large scale. Since the infrastructures systems are automated and computer-controlled, the industry informatization also brings security concerns. The recent Ukrainian power grid attack, the massive blackout in south American countries~\cite{haes2019survey} demonstrated that influence not only affects people's livelihoods but even international politics. The cyberattack on the ICS system can even cause physical damage to the infrastructure~\cite{zeller2011myth}, which makes it harder to recover. Incidents such as the Stuxnet proved this point. 

Consequently, ICS has received considerable attention due to security concerns. There are many ways to breach a computer system, and most of them are focus on software-based approaches such as vulnerability hunting and exploiting, cracking of authentication and protocols. Therefore, the attacker needs to break into the network and avoid existing access control and other mitigations. Moreover, most critical infrastructures use their air-gapped network, and It may take a state-sponsored team to accomplish such as mission ~\cite{langner2011stuxnet}. 

Under such circumstances, we believe that cyber-attacks with the assistance of physical approaches are underestimated, especially before the emergence of the so-called supply-chain attacks. Among those world-class attacks, the term APT(Advanced Persistent Threat) is often mentioned. In essence, the APT attack is an extremely well-hidden trojan that can be deployed for many years without being detected. Thus, installing the trojan and remotely triggering it is the crucial point of a successful attack.

% Importance of PLCs and attacks on them
The industrial control system (ICS) interconnects and controls the physical production assets. Compared to traditional IT infrastructures, the physical assets and the computer-based network's interconnections are a unique ICS feature. These interconnections are managed by embedded systems known as the programmable logic controller (PLC). Since, the PLCs are the core controllers for ICS, In recent years, several worldwide incidents, such as Stuxnet \cite{langner2011stuxnet} have targeted PLCs to attack the ICS and sabotage physical facilities. Several other attacks such as BlackEnergy \cite{cherepanov2016blackenergy, case2016analysis, soltan2016power, zhang2013time, williams2016power} targeted power grid critical infrastructures operated by ICS. Since, the power grids are more distributed and connected, any distributed coordinated attack on such infrastructure will lead to devastating results.

% Insufficiency of security current solutions

With the continuous emergence of such attacks, protection measures have also been strengthened. ICS security has been traditionally handling using network security practices such as access control. However, recent works have shown that such a traditional access control alone is not sufficient to prevent such attacks\cite{etigowni2016cpac}. A very common strategy is to use an isolated network from the Internet. However, this is not sufficient since several attacks that penetrated the air-gapped network \cite{cherepanov2016blackenergy, langner2011stuxnet, di2018triton}. Most of these defense mechanisms are driven by the NERC-CIP standards. However, several attacks targeting power grids have shown these regulations are not sufficient for a novel class of attacks \cite{huang2018algorithmic,garcia2017hey}.  

%Through physical access control, less software service is exposed to the public, reducing the potential attack surface, especially vulnerability-exploit-based attacks. However, real-world APT attacks show that even an air-gapped network is penetratable~\cite{langner2011stuxnet}. 

%firmware backdoors
The core of an APT attack is essentially a trojan backdoor that gains persistent access to a computer network and remains undetected for an extended period of time. The most crucial feature of a trojan is its stealthiness. To stay undetected over extended periods within the device is an essential issue that sophisticated APT attacks should consider. To achieve this, several works have utilized firmware modification techniques \cite{garcia2017hey, newman2011scada, basnight2013firmware, blanco2012one, cui2013firmware, konstantinou2015impact, schulz2017nexmon}. They inject malicious code into the target PLC, changing the working logic that runs in the device. However, such attacks are subjected to the firmware verification \cite{mcminn2012firmware, wang2015confirm, lee2016binding, li2011viper, seshadri2004swatt, li2010sbap} and update authentication \cite{lee2017blockchain, moran2019firmware, choi2016secure} method. It would be much harder for the attacker to implant the malicious code once the firmware update is encrypted and digitally-signed and the system applies the methods mentioned above.

Another critical point for the APT attack is choosing the trigger event. Usually, the PLC has a dedicated real-time microcontroller to control the physical world through its IO pins. However, the microcontroller does not directly communicate with the host, the central control terminal (human-machine interface, HMI). Therefore, firmware modification attacks can perform a preset task individually, but it is difficult to react to PLC firmware updates or coordinate a distributed attack with other controlled nodes, especially among air-gapped networks.

%Importance of HardDoor

Therefore, we propose an alternative approach to circumvent such existing software mitigations, \name, a parasitical hardware implant inside a PLC attaching to its circuit board and remotely controlled through GSM network.  With the recent emerging concept of supply chain attacks and real-world incidents \cite{oxfordsolarwinds}, such a hardware attack appears to be more practical. The hardware implant can be pre-installed during the PLC device's assembly line or even during the shipment \cite{robertson2018big}. Also, such implants can be stealthily installed on the site due to its vast distribution and low physical security throughout the grid \cite{Loopholes2020}.


We design it to be flexible because the PLC will load operating logic only after being deployed, and it is plausible that the ICS updates PLC's operating logic frequently. The hardware implant is specialized for the device's circuit board, the microcontroller, and other chips.  It controls the IO through the digital signal and bus-level protocol hijacking, independent of the PLC's firmware.

Memory bus and interconnect protocols such as SPI, I2C are all potential targets. Low-speed protocols are prevalent due to their simplicity. For instance, JTAG (Joint Test Action Group) is an industry-standard for verifying designs and testing integrated circuits (IC) after manufacture. On ARM microcontroller, extensive hardware features are also provided through this interface for system-level debugging and tracing. It can read/write registers of processor and memory during system runtime. We leverage this interface for IO controlling purposes, and also, we can fetch the PLC's firmware and operating logic for further offline analysis. Furthermore, with the ability to communicate through a GSM network, it is practical to control multiple nodes and organize a distributed attack simultaneously.


%Contributions

\textbf{Contributions.} To summarize, we make the following contributions in this paper:
\begin{itemize}[leftmargin=*]
	\item We present a novel attack class on industrial control systems: a parasitical hardware implant, which is completely invisible to the ICS control system.
	\item We disassembled and reverse engineered the circuit boards of a widely deployed Allen Bradley 1769-L18ER-BBIB CompactLogix 5370 PLC. 
	\item We develop a prototype implementation of \name, which is a small size device installed inside the Allen Bradley PLC. 
	\item We write a JTAG driver that runs bare-metally on a microcontroller with minimal resource usage.
	\item We test and evaluate \name and conduct a synchronized attack with multiple controlled Allen Bradley PLCs. 
\end{itemize}

%Roadmap

\textbf{Roadmap.} The rest of this paper is organized as the following. In~\autoref{sec:implant-background}, we provide the necessary background on programmable logic controllers and JTAG protocol. ~\autoref{sec:implant-overview} describes the objectives, adversary model and scope, challenges, and architecture of \name. ~\autoref{sec:implant-design} describes how we reverse-engineered the Allen Bradley PLC and prototyped \name with implementation details (~\autoref{sec:implant-implementation}) and evaluation (~\autoref{sec:implant-evaluation}), respectively. ~\autoref{sec:implant-relatedwork} provides a review of related work in the area of embedded system firmware attacks and mitigations. We also discuss our views on the hardware backdoor and the possible mitigation strategies in~\autoref{sec:implant-discussion}. Finally~\autoref{sec:implant-conclusions} concludes the paper.


