%! TEX root = 'main.tex'
\section{Related Work}
\label{sec:implant-relatedwork}



Firmware modification attacks~\cite{newman2011scada}~\cite{basnight2013firmware}~\cite{blanco2012one}~\cite{cui2013firmware}~\cite{konstantinou2015impact}~\cite{schulz2017nexmon} constitute significant attacks targeting embedded systems, industrial control systems, and IoT devices. 

Harvey~\cite{garcia2017hey} is a physical-aware stealthy rootkit against a cyber-physical power grid control system. It hides within the PLC's firmware below the control logic and modifies control commands before sending it to the physical plant's actuators. This work~\cite{moore2017implications} implements a malicious firmware that ignored incoming print commands for a printed 3D model, substitutes malicious print commands for an alternate 3D model.  If the firmware attacks can be carried out remotely, the harm will be even more significant. Cui et al.~\cite{cui2013firmware} gives a detailed case study of the HP-RFU (Remote Firmware Update) LaserJet printer firmware modification vulnerability, which allows arbitrary injection of malware into the printer's firmware via standard printed documents.

These systems have one thing in common: they all run on a microcontroller with limited computing power. Therefore, these devices run a simple real-time operating system, and there has not been a complete anti-virus system or a series of integrity verification features and programs provided by hardware and operating system like those on modern PCs. Due to the lack of security features, the programs run on those systems are often more vulnerable. Moreover, because they usually focus on specific areas and are not easily accessible, security issues are ignored. However, once those systems are compromised, through firmware modification, the attacker can stay in the dark for a long time without being discovered and cause a significant impact at a particular moment.

On the other hand, there are many ways to protect the firmware from being modified. ConFirm~\cite{wang2015confirm} is a low-cost technique to detect malicious modifications in the firmware of embedded control systems. It measures the number of low-level hardware events that occur during the execution of the firmware. Lee et al.~\cite{lee2016binding} presented a technique for binding software to hardware instances that use the devices' hardware security properties. The proposed technique assures manufacturers that only they can perform their hardware and software binding and create their products.

Remote software attestation~\cite{li2011viper} is also a defense against firmware modification.  SWATT~\cite{seshadri2004swatt} verifies embedded devices' memory contents and establishes the absence of malicious changes to the memory contents without using extra security hardware features. It uses a challenge-response protocol between the
verifier and the embedded device. The verifier sends a challenge to the embedded device. The embedded device computes a response to this challenge in a pre-defined protocol between the verifier and the device. The device can only give the correct answer if the memory content is intact. Otherwise, the attacker has to know the verifier's secret algorithm to break the verification. Similarly, SBAP~\cite{li2010sbap} also provides a software-only solution to verify the firmware integrity but with the help of an existing peripheral device.

Other methods such as firmware binary obfuscation~\cite{cyr2019low}~\cite{schrittwieser2016protecting}~\cite{cheng2019dynopvm},  makes it very challenging for firmware modification attacks. It requires comprehensively analyzing each device to find a suitable place to inject malicious code.
