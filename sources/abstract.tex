%! TEX root = 'main.tex'
\section{Abstract}
\label{sec:implant-abstract}

%Critical infrastructures such as power grids are under emerging advanced persistent threat (APT) attacks. Because the infrastructures systems are automated and computer-controlled, and the programmable Logic Controllers (PLCs) are the neurons that control the physical system. A stealthy backdoor usually is the core of an APT attack, which allows the attacker to launch remote malicious operations. However, to achieve further stealthiness and bypass exited software mitigations, it needs to evolve from high-level software into low-level hardware.
%
%This paper presents a small-size parasitical hardware implant attached to a victim Programmable Logic Controller (PLC). It controls the PLC by modifying the digital signal or hijacking the various buses on the boards. This attack can be deployed either during the supply chain or stealthily installed in remote plants. The personnel who carry out the installation do not require domain-specific knowledge of PLC communication protocols or existed software mitigations. The hardware implant contains a cellular chip that provides the remote control channel, which allows the attacker to organize a multi-point distributed attack.
%
%We evaluate the hardware implant with state-of-the-art software security measures, and the result shows that the hardware side attack will not violate integrity checks and gains complete control of the target devices. 


Critical infrastructures such as power grids are under emerging advanced persistent threat (APT) attacks. Because the infrastructures systems are automated and computer-controlled, and the programmable Logic Controllers (PLCs) are the neurons that control the physical system. A stealthy backdoor usually is the core of an APT attack, which allows the attacker to launch remote malicious operations. However, to achieve further stealthiness and bypass exited software mitigations, it needs to evolve from high-level software into low-level hardware.

This paper presents a small-size parasitical hardware implant attached to a victim Programmable Logic Controller (PLC). It controls the PLC by modifying the digital signal or hijacking the various buses on the boards. This attack can be deployed either during the supply chain or stealthily installed in remote plants. The hardware implant contains a cellular chip that provides the remote control channel, which allows the attacker to organize a multi-point distributed attack.

We evaluate the hardware implant with state-of-the-art software security measures, and the result shows that the hardware side attack will not violate integrity checks and gains complete control of the target devices. 


