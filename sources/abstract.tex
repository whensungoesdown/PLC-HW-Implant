%! TEX root = 'main.tex'
\section{Abstract}
\label{sec:implant-abstract}


Critical infrastructure such as the power grid is vital to national security.
Their failure or incapacity would have a significant impact on people's daily life on a large scale. However, they are under emerging advanced persistent threat (APT) attacks since the infrastructures systems are automated and computer-controlled. The programmable logic controllers (PLCs) are the neurons that control the physical system. In most APT attacks, a stealthy backdoor usually is the core that allows the attacker to hide in the dark without being detected and launch remote malicious operations at a particular moment. However, to achieve further stealthiness and bypass exited software mitigations, it needs to evolve from high-level software into low-level hardware.

This paper presents \name, a small-size parasitical hardware implant that attaches to the PLC's circuit board. It controls the PLC by modifying the digital signal or hijacking the various buses on the boards. This attack can be deployed either during the supply chain or stealthily installed in remote plants. The hardware implant contains a cellular chip that provides a remote control channel, which allows the attacker to organize a multi-point distributed attack.

We implement and evaluate \name with widely deployed Allen Bradley PLCs. The result shows that such a hardware backdoor does not change the firmware or induce overhead to the system, thus no integrity violation. It can secretly change the PLC's output without showing any trace. Furthermore, the attacker can even penetrate air-gapped networks communicating with \name and conduct a simultaneous attack with multiple controlled nodes.

