%! TEX root = 'main.tex'
\section{Abstract}
\label{sec:abstract}
Advanced persistent threat (APT) is emerging threat that targeting servers and critical infrastructures.
Its core units commonly includes a stealthy backdoor. Backdoor techniques have various styles and has evolved from high-level software into low-level hardware. Comparing to software attacks, hardware backdoor is more difficult to spot due to the lack of previous examples of exposure. Silicon level attacks such as changing chip design is stealthier, but it needs planning ahead for many years. Circuit board level attacks, on the other hand, is still practical even after field deployment. The circuit board modification can be disguised as parts that was originally belonged to the device. 

In this paper, we show how an after-deploy attack, performed by a personnel who have no domain-specific knowledge is feasible.We use extra circuit board that attached to the target device boards to interpret signals and to inject commands. A cellular chip was integrated to provide a remote control channel, which allows the attacker to organize distributed attacks simultaneously. The hardware implant will not trigger any existing firmware integrity inspection because the signal injection is done either in circuit level or through exposed interface on the board. We implement this attack in an Allen Bradley PLC and assemble a implant device. Experimental results show that our attacks work in real world scenario, show how several carefully selected target devices can bypass system security settings and compromise the power grid.
