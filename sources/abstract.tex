%! TEX root = 'main.tex'
\section{Abstract}
\label{sec:implant-abstract}


Critical infrastructure such as the power grid is vital to national security.
Their failure or incapacity would have a significant impact on people's daily life on a large scale. However, they are under emerging advanced persistent threat (APT) attacks since the infrastructures systems are automated and computer-controlled. The programmable logic controllers (PLCs) are the neurons that control the physical system. In most APT attacks, usually a stealthy backdoor is the core that allows the attacker to hide in the dark without being detected and launch remote malicious operations at a particular moment. However, to achieve further stealthiness and bypass existing software mitigations, it needs to evolve from high-level software into low-level hardware.

This paper presents \name, a small-size parasitical hardware implant that attaches to the PLC's circuit board. Using \name, the attacker can control the PLC remotely by hijacking the various buses on the boards and modifying the digital signal. This attack can be deployed either during the supply chain or stealthily installed in remote plants. The hardware implant contains a cellular chip that provides a remote control channel, that allows the attacker to organize a multi-point distributed attack by controlling several PLCs simultaneously.

We implement and evaluate \name on a popular and widely deployed Allen Bradley PLCs. The results show that such a hardware backdoor does not change the firmware, thus no integrity violation. \name also induces almost no overhead to the system, thus not affecting the runtime of the PLC. It can secretly change the PLC's outputs without leaving any trace. Furthermore, the attacker can even penetrate air-gapped networks communicating with \name and conduct a simultaneous attack with multiple controlled nodes.

