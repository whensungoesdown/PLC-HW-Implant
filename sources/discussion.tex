%! TEX root = 'main.tex'
\section{Discussions and Mitigations}
\label{sec:implant-discussion}

Reducing the size of the hardware implant is necessary but not the main factor in disguising. A customized PCB board allows all the required chips to be installed together so that there is no Dupont jumper wire, which makes the hardware implant very suspicious. The SIM800C chip already is a SiP (System in a Package), and the JTAG driver can run on a minimal core such as cortex-M0. Combining these two can make the backdoor simpler, and we can already find such WIFI chips, such as ESP32~\cite{pravalika2019internet}, as of the time of writing this paper. Moreover, a better camouflage makes the hardware backdoor look like a legit accessory of the PLC. The IPMI (Intelligent Platform Management Interface) module is a good example. It needs to be purchased separately and installed on the reserved socket on the server motherboard to activate the remote control function.

To mitigate a hardware backdoor attack such as \name, we first measure the power usage to find the overhead caused by the extra circuitry, as evaluated in ~\autoref{implant-evaluation}. The result shows a few current peaks when \name connects to the GSM network and sends JTAG commands to PLC. However, its power consumed is not much, a few milliamperes. Moreover, the PLC's power consumption is also constantly fluctuating, and setting a threshold with little redundancy will affect the system's reliability.

In particular, to avoid malicious use of the JTAG interface, the manufacturer can blow the corresponding physical JTAG fuse at the factory~\cite{rosenfeld2010attacks}~\cite{buskey2006protected}. Blowing the fuse completely disables the JTAG port and is not reversible.

However, by tapping the open wire on the boards, we can directly control each peripheral device because they are connected to the microcontroller through various buses. We think that some physical protection will cause trouble to the attacker. For example, the Chip-on-Board (COB)~\cite{lau1994chip} packaging with a black blob for low-cost IC items makes it more challenging to identify the chip underneath. Nevertheless, we believe it is not a good practice to cover the whole board, and heat dissipation is critical for large area ICs. One possible direction for preventing bus signal hijacking is to send package-based encrypted data, which requires the microcontroller and peripherals to exchange encryption keys and maintain a connection state. However, it may not be practical for low-speed devices and low-end microcontrollers.

